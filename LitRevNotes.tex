\cite{leibniz18591684, leibniz12supplementum, leibniz1863geometria} These are Leibniz's three papers where he \textcolor{red}{solely and originally} developed the theory of calculus and by extension differential equations. Important for the history of SIR models because it marks development of the relevant mathematical theory.

\cite{bernoulli1695explicationes} On a similar note to the above this seems to be the first time someone actually published a differential equation. 

\cite{bernoulli2004attempt} Continues the Bernoulli family contributions to the study of infectious diseases by using differential equations to investigate smallpox.

\cite{farr1866cattle, lilienfeld2007celebration, evans1876some} William Farr appears to have the distinction of being the first person to (publicly) predict the course of an ongoing outbreak.

\cite{BROWNLEE1915125, BrownleeJohn1916OTCO} John Brownlee endeavoured to deduce \textit{a priori} a general mathematical description for the course of an epidemic

\cite{ross1916application, ross1917application, RonaldRoss1917AAot} Ross and Hudson apply the theory of differential equations and arrive at something very similar to the modern form of the differential equations but did not employ the notion of compartmental models directly.

\cite{kermack1927contribution, brauer2005kermack} Kermack and Mckendrick proposed the first recognizable "SIR model".

\cite{hethcote2000mathematics} Gives a thorough survey of SIR models and generalizations thereof during the course of the 20th century.

\cite{reveller1969optimization, capasso1978generalization, satsuma2004extending} All make notable attempts to generalize the basic "SIR model" for various reasons.

\cite{walter1983some, walter1984eigenvalues, walter1985complex} Walter deals with "compartmental models" in the language of graph theory however the class of models that he studies are much more restrictive than commonly used in epidemiology. 

\cite{rescigno1960synthesis, rescigno1962analysis, rescigno1964some, rescigno1965some, rescigno1999compartmental, rescigno2001rise} Aldo Rescigno has published extensively on compartmental models as they are used to model biological systems however he makes similar assumptions to GG Walter with regards to linearity. Never-the-less his work is an important contribution towards developing a rigorous mathematical theory of biologically inspired compartmental models.

\cite{hearon1963theorems, hearon1972path, hearon1972residence, hearon1979monotonicity, hearon1981residence} JZ Hearon proceeds much in the same manner as A. Rescigno although he does at least acknowledge the existence of nonlinear models particularly in \cite{hearon1963theorems}.

\cite{berman1956invariants, berman1963formulation, berman1963postulate} again investigation focuses almost exclusively on linear systems.

\cite{savageau1988introduction, voit1988recasting, voit1990s} Represent a notable attempt at creating a general formalism for non-linear systems. In particular by recasting the underlying differential equation of a model into a canonical form they call an "S-model".


\cite{friston2020dynamic, fields2021age, chang2022stochastic, lavielle2020extension} All provide modified "SIR models" related to the COVID-19 pandemic.

\cite{kryazhimskiy2007state, gog2002dynamics} Discuss various approaches to multi-strain models.




