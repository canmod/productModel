\abstract{Compartmental models are valuable tools for investigating infectious diseases. Researchers building such models typically begin with a simple structure where compartments correspond to individuals with different epidemiological statuses, e.g., the classic SIR model which splits the population into susceptible, infected, and recovered compartments. However,
  as more information about a specific pathogen and/or the host is discovered, or as a means to investigate the effects of heterogeneities, it becomes useful to stratify models further --- for example by socio-demographics, immunity history, or pathogen strain. The operation of constructing stratified compartmental models from a pair of simpler models resembles the Cartesian product used in graph theory, but several key differences complicate matters. In this article we give explicit mathematical definitions for several so-called ``model products'' and provide examples where each is suitable. We also provide examples of model stratification where no existing model product will generate the desired result. }
